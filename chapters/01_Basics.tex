\section{Basics}

\paragraph{Notations}
\begin{itemize}
  \item $ \left( \begin{smallmatrix}
    V \\ k
  \end{smallmatrix} \right) \coloneqq \{ A : A \subseteq V \wedge \vert A \vert = k \} $
  \item $ [n] \coloneqq \{ 1, \dots, n \} \subset \N $
  \item \textbf{Power set} $ 2^X \coloneqq \{ A : A \subseteq X \} $
\end{itemize}

\paragraph{Graphs}
\begin{itemize}
  \item \textbf{Definition}: $ G = (V,E) $ with  $ E \subseteq V^2 $, $ V \cap E = \varnothing $
  \item \textbf{Vertex}: $ v \in V $ for graph $ G = (V,E) $
  \begin{itemize}
    \item $ v $ \emph{incident with} $ e \Leftrightarrow v \in e $  
    \item $ v_1 $, $ v_2 $ \emph{ends of} $ e \Leftrightarrow e = v_1v_2 $
    \item $ v_1 $, $ v_2 $ \emph{adjacent}/\emph{neighbors} $ \Leftrightarrow $ $ v_1v_2 \in E $
  \end{itemize}
  \item \textbf{Edge}: $ e = \{ x,y \} \in E $ for graph $ G = (V,E) $ (short $ e = xy $)
  \begin{itemize}
    \item $ e $ \emph{edge at} $ v \Leftrightarrow v $ incident with $ e $
    \item $ e $ \emph{joins} $ v_1 $, $ v_2 \Leftrightarrow e = v_1v_2 $
    \item $ xy $ is \emph{X-Y-edge} $ \Leftrightarrow x \in X \wedge y \in Y $
    \item $ e_1 $, $ e_2 $ \emph{adjacent}/\emph{neighbors} $ \Leftrightarrow \ \exists \ v: v \in e_1 \wedge v \in e_2 $
  \end{itemize}
  \item \textbf{Vertex sets}:
  \begin{itemize}
    \item $ V(G) = V $ for graph $ G = (V,E) $
    \item $ X \subset V(G) $ \emph{independent} $ \Leftrightarrow $ no $ x_1, x_2 \in X $ are adjacent
    \item \emph{neighborhood} of $ v \in V(G) $: $ N(v) = \{ u \in V(G) : uv \in E(G) \} $
  \end{itemize}
  \item \textbf{Edge sets}:
  \begin{itemize}
    \item $ E(G) = E $ for graph $ G = (V,E) $
    \item $ E(X,Y) $: set of edges between $ X \subset V(G) $ and $ Y \subset V(G) $
    \item $ E(x,Y) $: set of edges between vertex $ x \in V(G) $ and $ Y \subset V(G) $ 
    \item $ E(v) $: set of edges at $ v \in V(G) $
  \end{itemize}
  \item \textbf{Order}: $ = \vert V(G) \vert $, short $ \vert G \vert $
  \item \textbf{Size}: $ = \vert E(G) \vert $, short $ \Vert G \Vert $
  \item \textbf{Trivial graph}: graph of order $ 0 $ or $ 1 $
  \item \textbf{Isomorphic} ($ G_1 $ to another graph $ G_2 $, write $ G_1 \cong G_2 $ or even $ G_1 = G_2 $):
    \begin{equation*}
      \exists \text{ bijection } f: V_1 \to V_2 : \{ u,v \} \in E_1 \Leftrightarrow \{ f(u),f(v) \} \in E_2
    \end{equation*}
  \item \textbf{Graph union}: $ G \cup G' = (V(G) \cup V(G'), E(G) \cup E(G')) $
  \item \textbf{Graph intersection}: $ G \cap G' = (V(G) \cap V(G'), E(G) \cap E(G')) $
  \item \textbf{Graph multiplication}: $ G * G' $: join all $ v \in G $ with all $ v' \in G' $ \\* (with $ V(G) \cap V(G') = \varnothing $)
  \item \textbf{Subgraph} $ G' $ of $ G $ (write $ G' \subseteq G $): if $ V(G') \subseteq V(G) $ and $ E(G') \subseteq E(G) $
  \begin{itemize}
    \item $ G $ \emph{contains} $ G' $ 
    \item $ G' $ \emph{proper subgraph} of $ G $: if $ G' \subseteq G $ and $ G' \neq G $
    \item $ G' $ \emph{induced subgraph} of $ G $: $ G' \subseteq G $ and $ E(G') $ contains all edges of $ G $ with both ends in $ V(G') $, $ V(G') $ \emph{induces} $ G' $, write $ G' = G[X] $ (with $ X = V(G') $)
    \item \emph{Edge-induced subgraph}: subgraph induced by $ X \subseteq E(G) $, note $ G[X] $
    \item $ G' $ \emph{spanning subgraph} of $ G $: $ V(G') = V(G) $
  \end{itemize}
  \item \textbf{Supergraph}: $ G $ of $ G' $ (write $ G \supseteq G' $): as above.
  \item \textbf{Vertex cover}: $ V' \subseteq V(G) $ s.t. any $ e \in E(G) $ is incident to a vertex in $ V' $ 
  \item \textbf{Graph subtraction}:
  \begin{itemize}
    \item $ G - U = G[V(G) \setminus U] $ for some vertex set $ U $
    \item $ G - v = G[V(G) \setminus \{ v \}] $ for some vertex $ v $ 
    \item $ G - G' = G[V(G) \setminus V(G')] $ for some graph $ G' $
  \end{itemize}
  \item \textbf{Edge addition}: $ G + F = (V(G), V(E) \cup F) $ for some $ F \subseteq V(G)^2 $
  \item \textbf{Complement}: $ \overline{G} = (V(G), V^2 \setminus E(G)) $
  \item \textbf{Line graph} of $ G $: $ L(G) = (E(G), \{ xy \in E(G)^2 : x,y \text{ adjacent in } G \}) $ 
  \item \textbf{Complete graph}: $ (X, X^2) $ with vertex set $ X $
  \begin{itemize}
    \item $ K_n $: complete graph on $ n $ vertices 
  \end{itemize}
\end{itemize}

\paragraph{Vertex degrees}
\begin{itemize}
  \item \textbf{Degree} of $ v \in V $: $ d(v) = \deg(v) = \vert N(v) \vert $
  \begin{itemize}
    \item $ v \in V(G) $ \emph{isolated}: $ d(v) = 0 $
    \item $ v \in V(G) $ \emph{leaf}: $ d(v) = 1 $
    \item number of vertices of odd degree is even
  \end{itemize}
  \item \textbf{Minimum degree} of graph $ G $: $ \delta(G) = \min\{ d(v) : v \in V(G) \} $
  \item \textbf{Maximum degree} of graph $ G $: $ \Delta(G) = \max\{ d(v) : v \in V(G) \} $
  \item \textbf{Degree sum}: $ \sum_{v \in V(G)} \deg(v) = 2\vert E(G) \vert $
  \item \textbf{Average degree} of graph $ G $: $ d(G) = \frac{1}{\vert G \vert}\sum_{v\in V} d(v) $
  \begin{itemize}
    \item $ \delta(G) \leq d(G) \leq \Delta(G) $ 
  \end{itemize}
  \item \textbf{k-regular graph}: $ \forall v \in V(G) : d(v) = k $
  \begin{itemize}
    \item \emph{cubic graph}: $ 3 $-regular graph 
  \end{itemize}
  \item \textbf{Vertex-Edge-ratio} of graph $ G $: $ \epsilon(G) = \frac{\Vert G \Vert}{\vert G \vert} $
  \begin{itemize}
    \item $ \epsilon(G) = \frac{1}{2}d(G) $
    \item every graph with $ \Vert G \Vert \geq 1 $ has $ H \subseteq G $ with $ \delta(H) > \epsilon(H) \geq \epsilon(G) $
  \end{itemize}
\end{itemize}

\paragraph{Paths}
\begin{itemize}
  \item \textbf{Path}: $ (\{ v_1, \dots, v_n \}, \{ \{ v_1, v_2 \}, \dots, \{ v_{n-1},v_n \} \}) $ (read: $ v_0v_n $-path)
  \begin{itemize}
    \item \emph{shorthand}: $ v_1\dots v_n $
    \item $ v_0 $, $ v_n $ \emph{linked} by path 
    \item $ v_0 $, $ v_n $ \emph{end-vertices}/\emph{ends} of path
    \item $ v_1, \dots, v_{n-1} $ \emph{inner vertices} of path
  \end{itemize}
  \item \textbf{Length}: $ \vert E(P) \vert \neq \vert V(P) \vert $
  \item \textbf{Shorthands} ($ 0 \leq i \leq j \leq k $):
  \begin{itemize}
    \item $ P = x_0\dots x_k $, $ \mathring{P} = x_1\dots x_{k-1} $
    \item $ Px_i = x_0\dots x_i $, $ P\mathring{x_i} = x_0\dots x_{i-1} $
    \item $ x_iP = x_i\dots x_k $, $ \mathring{x_i}P = x_{i+1}\dots x_k $
    \item $ x_iPx_j = x_i\dots x_j $, $ \mathring{x_i}P\mathring{x_j} = x_{i+1}\dots x_{j-1} $
  \end{itemize}
  \item \textbf{Path concatenation}: $ Px \cap xQy \cap yR = PxQyR $
  \item \textbf{A-B-path}: $ V(P) \cap A = \{ x_0 \} \wedge V(P) \cap B = \{ x_n \} $
  \item \textbf{H-path}: graph $ H $, $ P $ meets $ H $ exactly in its ends

  % TODO this has not been disclosed yet :)
  \item \textbf{Independent}: two $ ab $-paths are independent $ \Leftrightarrow $ they only share $ a $ and $ b $
  \item \textbf{Path existence}: Every $ G $ with $ \delta(G) \geq 2 $ contains path of length $ \delta(G) $
  \item \textbf{Distance}: $ d_G(x,y) = \min\left( \{ k : \ \exists \text{ $ x $-$ y $-path of length } k \} \cup \{ \infty \} \right) $
  \item \textbf{Central}: $ v \in V(G) $ where $ \text{cen} = \max\{ d_G(v,x) : v \neq x \in V(G) \} $ is minimal
  \item \textbf{Radius}: $ \text{rad}(G) = $ minimal cen $ = \min_{x \in V(G)}\max_{y \in V(G)}d_G(x,y) $
  \item \textbf{Diameter} of $ G $: $ \text{diam}(G) = \max\{ d_G(x,y) : x,y \in V(G) \} $
  \begin{itemize}
    \item \emph{radius-diameter-relation}: $ \text{rad}(G) \leq \text{diam}(G) \leq 2\text{rad}(G) $
    \item \emph{radius-degree-vertex-restriction}:
    \begin{equation*}
      \text{rad}(G) \leq k \wedge \Delta(G) \leq d \geq 3 \Rightarrow \vert G \vert \leq \frac{d}{d-2}(d-1)^k
    \end{equation*}
  \end{itemize}
  \item \textbf{Walk}: alternating sequence $ v_0e_0\dots e_{k-1}v_k $ s.t. $ e_i = v_iv_{i+1} $ ($ \forall i < k $)
  \begin{itemize}
    \item \emph{closed walk}: $ v_k = v_0 $ 
    \item \emph{walk-path-relation}: all vertices in walk distinct $ \leadsto $ path
    \item \emph{walk-path-induction}: $ \exists \ v_0v_k $-walk $ \Rightarrow \ \exists \ v_0v_k $-path 
  \end{itemize}
\end{itemize}

\paragraph{Cycles}
\begin{itemize}
  \item \textbf{Cycle}: $ C = P + x_{k-1}x_0 $ with path $ P = x_0\dots x_{k-1} $ ($ k \geq 3 $)
  \begin{itemize}
    \item \emph{shorthand}: $ x_0\dots x_{k-1}x_0 $ 
  \end{itemize}
  \item \textbf{Length}: $ = \vert C \vert = \Vert C \Vert $
  \item \textbf{k-cycle}: $ C_k = $ cycle of length $ k $
  \item \textbf{Girth} of graph $ G $: $ g(G) = \min\left( \{ k : G \text{ contains } C_k \} \cup \{ \infty \} \right) $
  \begin{itemize}
    \item \emph{girth-diameter-relation}: $ g(G) \leq 2\text{diam}(G)+1 $ 
    \item \emph{girth-vertex-relation}: $ \delta(G) \geq 3 \Rightarrow g(G) < 2\log \vert G \vert $
  \end{itemize}
  \item \textbf{Circumference} of graph $ G $: $ = \max\left( \{ k : G \text{ contains } C_k \} \cup \{ 0 \} \right) $
  \item \textbf{Chord} of cycle $ C \subseteq G $: $ = xy \in E(G) $ with $ xy \not \in E(C) $, but $ x,y \in V(C) $
  \item \textbf{Induced cycle}: induced subgraph of $ G $ that is a cycle (= cycle in $ G $ with no chords)
  \item \textbf{Cycle existence}: Every $ G $ with $ \delta(G) \geq 2 $ contains cycle of length $ \geq \delta(G) + 1 $
  \item \textbf{Odd closed walk, odd cycle}: $ G $  has \emph{odd} closed walk $ \Rightarrow $ $ G $ has odd cycle
\end{itemize}

\paragraph{Connectivity}
\begin{itemize}
  \item \textbf{Connected} graph $ G $: $ \forall x,y \in V(G) : \exists \ xy $-path
  \begin{itemize}
    \item \emph{connected subset} $ U \subseteq V(G) $: if $ G[U] $ is connected  
  \end{itemize}
  \item \textbf{Vertex enumeration}: $ G $ connected $ \Rightarrow $ vertices can be enumerated $ v_1, \dots, v_n $ s.t. $ G_i \coloneqq G[v_1, \dots, v_i] $ is connected ($ \forall i \leq n $)
  \item \textbf{Component}: maximal connected subgraph
  \begin{itemize}
    \item \emph{graph partitioning}: components partition $ G $ 
  \end{itemize}
  \item \textbf{Subgraph separation}: $ X \subset V(G) $ separates $ A,B \subset V(G) \Leftrightarrow $ any $ A $-$ B $-path has vertex in $ X $
  \item \emph{separator} $ X $
  \item \textbf{Cut-Vertex}: vertex separating two other vertices of the component
  \item \textbf{Bridge}: edge separating its ends (= edges of component not lying on any cycle)
  \item \textbf{k-connected}: if $ \vert G \vert > k \wedge G - X $ is connected $ \forall X \subseteq V(G) $ with $ \vert X \vert < k $
  \begin{itemize}
    \item[$ \leadsto $] no two vertices in $ G $ are separated by fewer than $ k $ other vertices
  \end{itemize}
  \item \textbf{Connectivity}: $ \kappa(G) = \max\{ k : G \text{ is $ k $-connected} \} $
  \item \textbf{l-edge-connected}: if $ \vert G \vert > 1 \wedge G - F $ is connected $ \forall F \subseteq E(G) $ with $ \vert F \vert < l $
  \item \textbf{Edge-connectivity}: $ \kappa'(G) = \lambda(G) = \max\{ l : G \text{ is $ l $-edge-connected} \} $
  \item \textbf{Connectivity and smallest degree}: $ \kappa(G) \leq \kappa'(G) \leq \delta(G) $
  \item \textbf{Connectivity and average degree}: $ d(G) \geq 4k \Rightarrow G $ has $ k $-connected subgraph
\end{itemize}

\paragraph{Rest}
\begin{itemize}
  \item \textbf{Degree sequence}: multiset of degrees of vertices in $ V(G) $
  \begin{itemize}
    \item \emph{graphic}: deg. seq. $ (d_1, \dots, d_n) $, iff
    \begin{enumerate}
       \item $ d_1 + \cdots + d_n $ even
       \item $ \sum_{i=1}^k d_i \leq k(k-1) + \sum{i = k+1}^n \min(d_i,k) $ \quad ($ \forall 1 \leq k \leq n $)
     \end{enumerate} 
  \end{itemize}
  \item \textbf{Adjacency matrix}: $ A(G) = \R^{n \times n} \ni A_{i,j} = \begin{cases}
    1, & ij \in E \\
    0, & \text{else}
  \end{cases} $
  \item \textbf{Incidence graph} of $ G $: $ IG = (V \cup E, \left\{ \{ v,e \} : v \in e, e \in E \right\}) $
  \item \textbf{Eulerian}: if it contains an Eulerian tour
  \item \textbf{Connected}: for any two vertices there is a link between them
  \begin{itemize}
    \item \emph{spanning tree}: if $ G $ is connected, then it has a spanning tree
    \item \emph{peeling leaves}: vertices can be ordered $ v_1, \dots, v_n $ s.t. $ G[\{ v_1, \dots, v_i \}] $ is connected for $ i \in \{ 1, \dots, n \} $
  \end{itemize}
\end{itemize}

\paragraph{Digraph}
\begin{itemize}
  \item \textbf{Definition}: $ G = (V,E) $ with vertex set $ V $ and edge set $ E \subseteq \{ (u,v) : u,v \in V, u \neq v \} $
\end{itemize}

\paragraph{Multigraph}
\begin{itemize}
  \item \textbf{Definition}: $ G = (V,E) $ with vertex set $ V $ and multiset $ E $ of $ V $-pairs
\end{itemize}

\paragraph{Hypergraph}
\begin{itemize}
  \item \textbf{Definition}: $ G = (V,E) $ with vertex set $ V $ and edge set $ E \subseteq 2^V = \{ A : A \subseteq V \} $
\end{itemize}

\paragraph{Walk}
\begin{itemize}
  \item \textbf{Definition}: non-empty alternating sequence of vertices and edges
    \begin{equation*}
      v_0e_0\dots e_{k-1}v_k
    \end{equation*}
    with $ e_i = v_iv_{i+1} $, length $ k \in \N $
    \begin{itemize}
      \item \emph{closed}: if $ v_0=v_k $
      \item \emph{even}: if $ k $ is even
      \item \emph{odd}: if $ k $ is odd
    \end{itemize}
  \item \textbf{Eulerian tour}:
  \begin{itemize}
    \item \emph{Definition}: closed walk with
    \begin{itemize}
      \item no edges of $ G $ are repeatedly used
      \item all edges of $ G $ are used 
    \end{itemize}
    \item \emph{Even degrees}: $ G $ connected has Euler tour $ \Leftrightarrow \forall v \in V(G) : \deg(v) $ even
  \end{itemize}
\end{itemize}

\paragraph{Connected component}
\begin{itemize}
  \item \textbf{Definition}: \emph{maximal} connected subgraph (connected, but any supergraph isn't)
\end{itemize}

\paragraph{Block}
\begin{itemize}
  \item \textbf{Block}: maximal $ 2 $-connected subgraph or bridge
  \begin{itemize}
     \item share $ \leq 1 $ vertices with one another 
   \end{itemize} 
   \item \textbf{Block-cut-vertex graph}
   \begin{itemize}
     \item $ V = $ set of blocks $ \cup $ set of vertices
     \item $ E = \{ \{ v,B \} : v \in V(B) \text{, cut-vertex $ v $, block $ B $} \} $ 
     \item block-cut-vertex graph of connected graph is tree
   \end{itemize}
\end{itemize}

\paragraph{Acyclic graph, Forest}
\begin{itemize}
  \item \textbf{Definition}: Graph with no cycle as subgraph
\end{itemize}

\paragraph{Tree}
\begin{itemize}
  \item \textbf{Definition}: Graph that is connected and acyclic
  \begin{itemize}
    \item $ \Leftrightarrow G $ is connected and $ \forall e \in E(G): G-e $ is disconnected  \\*
      \phantom{$ \Leftrightarrow $} (\emph{minimal-connected})
    \item $ \Leftrightarrow G $ is acyclic and $ \forall xy \not \in E(G) : G \cup xy $ has cycle \\*
      \phantom{$ \Leftrightarrow $} (\emph{maximal-acyclic})
    \item $ \Leftrightarrow G $ is connected and \emph{1-degenerate} ($ \forall G' \subseteq G : \delta(G') \leq 1 $) 
    \item $ \Leftrightarrow G $ is connected and $ \Vert G \Vert = \vert G \vert - 1 $
    \item $ \Leftrightarrow G $ is acyclic and $ \Vert G \Vert = \vert G \vert - 1 $
    \item $ \Leftrightarrow \forall u,v \in V(G) \ \exists $ unique $ uv $-path
  \end{itemize}
  \item \textbf{Special trees}: path, star, spider, caterpillar, broom
  \item \textbf{Leaf existence}: Tree $ T $, $ \vert T \vert \geq 2 \Rightarrow T $ has leaf
  \item \textbf{Edge count}: Tree $ T $, $ \vert T \vert = n \Rightarrow \Vert T \Vert = n-1 $ 
\end{itemize}

% TODO MISSING KNESER GRAPH

% TODO MISSING PETERSON GRAPH

% TODO MISSING HYPERCUBE

\paragraph{Bipartite graph}
\begin{itemize}
  \item \textbf{Definition}: $ G $ is bipartite $ \Leftrightarrow G $ contains no cycles of odd length
  \begin{itemize}
    \item \emph{complete bipartite}: $ K_{m,n} = (A \cup B, \{ a,b \} : a \in A, b \in B) $ 
  \end{itemize}
  \item \textbf{Matchings}:
  \begin{itemize}
    \item \emph{saturating}: $ G = (A \cup B, E) $ has matching saturating $ A $ \\*
      \phantom{\emph{saturating}:} $\Leftrightarrow \forall S \subseteq A : N(S) \geq \vert S \vert $ \ \ \ ($ N(S) \coloneqq \{ b \in B : ab \in E, a \in S \} $)
    \item \emph{nearly}: $ G = (A \cup B, E) $, $ \forall S \subseteq A : \vert N(S) \vert \geq \vert S \vert - d $ \ \ \  ($ d \geq 1 $). \\*
      \phantom{\emph{nearly}:} $ \Rightarrow \ \exists $ matching $ M $ saturating all but at most $ d $ vertices of $ A $
  \end{itemize}
  \item \textbf{Matching vs vertex cover}: size of largest matching = size of smallest vertex cover
\end{itemize}

\paragraph{Matching}
\begin{itemize}
  \item \textbf{Definition}: graph with $ \delta(G) = \Delta(G) = 1 $ 
  \item \textbf{Perfect matching}: spanning + matching subgraph of $ G $ (aka \emph{1-factor})
  \begin{itemize}
    \item \emph{existence}: $ G $ has perfect matching $ \Leftrightarrow \forall S \subseteq V(G) : q(G-S) \leq S $ \\*
      ($ q(G) = $ number of components in $ G $ with odd order) 
  \end{itemize}
  % TODO STABLE MATCHINGS
\end{itemize}

\paragraph{Factors}
\begin{itemize}
  \item \textbf{k-factor}: spanning $ k $-regular subgraph (easy to find) 
  \item \textbf{f-factor}: spanning subgraph $ H \subseteq G $ with $ \deg_H(v) = f(v) $, \\*
  % TODO Kühn + Osthus on f-factors
    $ f: V(G) \to \{ 0,1,\dots \} $ with $ f(v) \leq \deg(v) $ \quad ($ \forall v \in V $)
  \item \textbf{H-factor} (aka \emph{perfect $ H $-packing}): spanning subgraph s.t. each component is $ \cong H $
  \begin{itemize}
    \item \emph{existence}: if $ \delta(G) \geq \left( 1 - \tfrac{1}{k}\vert V(G) \vert \right) $ and $ k $ divides $ \vert G \vert $, then $ G $ has $ K_k $-factor
  \end{itemize}
  % TODO f-factors in G <=> 1-factors in T(G,f)
\end{itemize}

\paragraph{Connectivity}
\begin{itemize}
  \item \textbf{$ k $-connected}: if $ \vert G \vert > k $ and deleting $ < k $ vertices does not disconnect $ G $
  \item \textbf{$ k $-linked}: if for any $ 2k $ vertices $ (s_1, \dots, s_k, t_1, \dots, t_k) \ \exists $ pairwise disjoint $ s_it_i $-paths (\emph{note}: $ k $-connected $ \not \Rightarrow k $-linked)
  % TODO Wollan 2005: 10k-connected => k-linked
  \item \textbf{Vertex-connectivity}: $ \kappa(G) = \max\{ k : \text{$ G $ is $ k $-connected} \} $
  \item \textbf{$ l $-edge-connected}: if deleting $ < l $ edges does not disconnect $ G $
  \item \textbf{Edge-connectivity}: $ \kappa'(G) = \max\{ l : \text{$ G $ is $ l $-edge-connected} \} $
  \item \textbf{Vertex- vs Edge-connectivity}: $ \kappa(G) \leq \kappa'(G) \leq \delta(G) $
  \item \textbf{Three-connected + contraction}: $ 3 $-connected $ \Leftrightarrow \ \exists $ separate $ G_0, \dots, G_k $ with
  \begin{equation*}
    G_0 = K_4, \ G_k = G, \ G_i = G_{i+1} \circ xy
  \end{equation*}
  with $ \deg(x), \deg(y) \geq 3 $
  \item \textbf{Three-connected + decontraction}: all $ 3 $-connected graphs can be built by iteratively de-contracting vertices of $ K_4 $
  \item \textbf{Average degree $ \geq 4 $}: has $ k $-connected subgraph ($ k \geq 2 $)
\end{itemize}

\paragraph{Cuts}
\begin{itemize}
  \item \textbf{Cut-Set}: $ X \subseteq V(G) \cup E(G) $ s.t. \#components in $ (G-X) $ greater than in $ G $ 
  \item \textbf{Cut-Vertex}: Cut-Set consisting of single vertex
  \item \textbf{Cut-Edge} (or \emph{bridge}): Cut-Set consisting of single edge
  % TODO bipartite edge-cuts (not found in script)
  \item \textbf{Menger's theorem}: for $ A,B \subseteq V(G) $:
    \begin{equation*}
      \text{min \# of vertices separating $ A $ and $ B $} = \text{max \# of disjoint $ A $-$ B $-paths}
    \end{equation*}
  \item \textbf{Menger global}:
    \begin{enumerate}
      \item \emph{$ k $-connected} $ \Leftrightarrow \forall a,b \in V(G) \ \exists \ k $ pairwise independent $ ab $-paths
      \item \emph{$ k $-edge-connected} $ \Leftrightarrow \forall a,b \in V(G) \ \exists \ k $ pairwise edge-disjoint $ ab $-paths 
    \end{enumerate}
\end{itemize}

\paragraph{Ear-decomposition}
\begin{itemize}
  \item \textbf{Definition}: $ G $ has \emph{ear-decomposition} $ \Leftrightarrow \ \exists $ sequence of graphs $ G_0, \dots, G_k $ with $ G_k = G $, $ G_0 = $ cycle, $ G_{i+1} $ obtained from $ G_i $ by attaching ``ear'' (path that shares only endpoints with $ G_i $)
  \item \textbf{2-connected} $ \Leftrightarrow \forall $ cycles $ C $ in $ G $ there is ear-decomposition starting at $ C $
\end{itemize}

\paragraph{Edge contraction}
\begin{itemize}
  \item \textbf{Contraction}:
  \begin{multline*}
    G \circ xy = ((V \setminus \{ x,y \}) \cup v_{xy}, \\
      (E \setminus \{ e: x \in E \vee y \in e \}) \cup \{ v_{xy}z : z \in (N_G(x) \cup N_G(y))\setminus \{ x,y \} \})
  \end{multline*}
  with $ xy \in E(G) $
  \item \textbf{De-contraction}: if $ \exists \ xy \in E(G) : \kappa(G \circ xy) \geq 3 $ \\*
    (for $ G $ with $ \kappa(G) \geq 3 $, $ \vert G \vert \geq 5 $) 
\end{itemize}

\paragraph{Planar graph tools}
\begin{itemize}
  \item \textbf{Homeomorphism}: $ f: \R^n \to \R^n $ continuous s.t. $ f^{-1} $ is also continuous 
  \item \textbf{Arc}: homeomorphic image of $ [0,1] $ in $ \R^2 $ under $ f $
  \begin{itemize}
    \item \emph{endpoints}: $ f(0) $ and $ f(1) $ $ \leadsto $ arc ``joins'' endpoints
    \item \emph{polynomial arc}: arc that is union of finitely many straight line segments
  \end{itemize}
  \item \textbf{Region} $ Y \subseteq \R^2 \setminus X $: any two points $ \in Y $ could be joined by arc and $ Y $ is maximal ($ X \subseteq \R^2 $)
  \item \textbf{Boundary} of $ X \subseteq \R^2 $:
    \begin{equation*}
      \delta X = \{ y \mid \forall \epsilon > 0 : B(y, \epsilon) \text{ contains points of $ X $ and not of $ X $} \}
    \end{equation*}
  \item \textbf{Jordan curve theorem}: If $ X \subseteq \R^2 $ and homeomorphic to $ \{ \overline{x} : \text{dist}(\overline{x}, 0) = 1 \} $ (\emph{unit circle}), then $ \R^2 \setminus X $ has two regions $ R_1 $, $ R_2 $ and $ \delta R_1 = X = \delta R_2 $.
\end{itemize}

\paragraph{Plane graph}
\begin{itemize}
  \item \textbf{Definition}: graph such that $ E(G) $ is set of arcs in $ \R^2 $ and endpoints of arcs in $ E(G) $ are vertices and:
  \begin{itemize}
    \item $ \forall e, e' \in E $, $ e \neq e' $: $ e $ and $ e' $ have distinct sets of edge sets
    \item $ \forall e \in E $, $ \mathring{e} = e \setminus \{ \text{endpoints} \} $ doesn't contain any vertices and points from other arcs 
  \end{itemize}
  \item \textbf{Faces}: regions of $ \R^2 \setminus \left( \bigcup_{e \in E} e \cup V \right) $
  \item \textbf{Maximally plane}: no edges can be added without breaking planarity
  \begin{itemize}
    \item \emph{plane triangulation}: every face is bounded by triangle $ \Leftrightarrow $ graph is maximally plane 
  \end{itemize}
  \item \textbf{Edge limitation 1}: Plane graph: $ \vert G \vert \geq 3 \Rightarrow \Vert G \Vert \leq 3n-6 $
  \item \textbf{Edge limitation 2}: Plane graph with no $ \triangle $: $ \Vert G \Vert \leq 2\vert G \vert - 4 $
  \item \textbf{Properties}: Let $ G $ be plane graph and $ H \subseteq G $.
  \begin{itemize}
    \item \emph{face inheritance}: $ \forall f \in F(G) \ \exists \ f' \in F(H) : f' \supseteq f $ 
    \item \emph{border inheritance}: $ \delta f \subseteq H \Rightarrow f' = f $
    \item \emph{edge-border relations}: $ e \in E(G), f \in F(G) \Rightarrow e \subseteq \delta f \vee \delta f \cap \mathring{e} = \varnothing $
    \item \emph{edges in circles}:
      \begin{align*}
        e \in E(G) \text{ is edge of a cycle } &\Rightarrow e \text{ is on boundary of exactly $ 2 $ faces} \\
          \text{ not edge of a cycle} &\Rightarrow e \text{ is on boundary of exactly $ 1 $ face}
      \end{align*}
      \item \emph{faces in cycles}: $ f_1, f_2 \in F(G) $. $ f_1 \neq f_2 \wedge \delta f_1 = \delta f_2 \Rightarrow G $ is cycle
      \item \emph{cyclic boundaries}: $ \kappa(G) \geq 2 \Rightarrow $ each face is bounded by cycle
    \item \emph{plane forests}: plane forests have exactly $ 1 $ face
  \end{itemize}
  \item \textbf{Dual multigraph}: Given plane $ G $:
  \begin{enumerate}
    \item Insert vertex in each face 
    \item Put edge $ \tilde{e} $ between vertices if respective faces share $ e $ (s.t. $ \tilde{e} $ and $ e $ cross once)
    \item \emph{Result}: Dual graph $ G' $ of $ G $ (plane multigraph)
  \end{enumerate}
  $ \leadsto $ faces of $ G $ properly $ k $-colored $ \Leftrightarrow \ \exists $ proper $ k $-coloring of vertices of $ G' $
\end{itemize}

\paragraph{Planar graph}
\begin{itemize}
  \item \textbf{Definition}: graph s.t. $ \exists $ plane graph $ G' $ and bijection $ f: V(G) \to V(G') $ s.t. $ \forall u,v \in V(G), \ uv \in E(G) : f(u), \ f(v) $ are endpoints of arc in $ G' $
  \item \textbf{Planar embedding} of $ G $: $ f $ from the definition
  \item \textbf{Planar because of minors}: The following statements are equivalent:
  \begin{itemize}
    \item $ G $ is planar 
    \item $ G \not \supseteq MK_5 \wedge G \not \supseteq MK_{3,3} $
    \item $ G \not \supseteq TK_5 \wedge G \not \supseteq TK_{3,3} $
  \end{itemize}
  \item \textbf{Euler's formula}: If $ G $ is connected plane graph with $ f $ faces, then
    \begin{equation*}
      \vert G \vert - \Vert G \Vert + f = 2
    \end{equation*}
  \item \textbf{$ \delta(G) $ limitation}: Planar graph $ \delta(G) \leq 5 $
  \item \textbf{Non-planar graphs}: $ K_5 $ and $ K_{3,3} $ are not planar
  \item \textbf{Kuratowski's lemmas}:
  \begin{enumerate}
    \item $ (TK_5 \subseteq G \vee TK_{3,3} \subseteq G) \Leftrightarrow MK_5 \subseteq G \vee MK_{3,3} \subseteq G $ 
    \item $ \kappa(G) \geq 3 \wedge MK_5 \not \subseteq G \wedge MK_{3,3} \not \subseteq G \Rightarrow G $ is planar
    \item $ \kappa(G) \geq 3 $, $ G $ edge-maximal wrt not containing $ TX $. If $ S $ is vertex-cut of $ G $, $ \vert S \vert \leq 2 \wedge G = G_1 \cup G_2 $, $ S = V(G_1) \cap V(G_2) $, then $ G_i $ is edge-maximal with no $ TX $ and $ S $ induces an edge
    \item $ \vert G \vert \geq 3 $, $ G $ edge-maximal wrt not containing $ TK_5 $ and $ TK_{3,3} \Rightarrow \kappa(G) \geq 3 $
  \end{enumerate}
  \item \textbf{2-cell} (embedding of $ G $ on surface $ S $): any closed simple curve in any region of $ S - G $ is continuously contractible into a point
  \item \textbf{Euler characteristic}: $ G $ embedded on surface $ S \Rightarrow n - e + f = $ \emph{Euler characteristic} is invariant
  \item \textbf{Euler genus}: $ n - e + f = 2 - 2 \gamma \leadsto $ \emph{Euler genus} $ 2 \gamma $ of $ S $
  \item \textbf{Heawood's formula}: $ \chi(G) \leq \underbrace{\left\lfloor \frac{7 + \sqrt{1 + 48 \gamma}}{2} \right\rfloor}_{f(\gamma)\text{, Heawoods number}} $ \\* (for $ G $ embedded on $ S $ with Euler char $ 2-2\gamma $)
  \item \textbf{Klein bottle}: $ K_{f(\gamma)} $ is embeddable on $ S $, unless $ S $ is \emph{klein bottle}
\end{itemize}

\paragraph{Minors}
\begin{itemize}
  \item \textbf{MH}: $ G \overset{(\star)}{=} MH $ is \emph{minor} of $ H $ if
  \begin{itemize}
    \item $ V(G) = V_1 \overset{\cdot}{\cup} \cdots \overset{\cdot}{\cup} V_n $ with $ n = \vert H \vert $
    \item $ G[V_i] $ connected ($ \forall i = 1, \dots, n $)
    \item If $ V(H) = \{ v_1, \dots, v_n \} $ and $ v_iv_j \in E(H) $, then $ \exists $ edge between $ V_i $ and $ V_j $
  \end{itemize}
  $ (\star) $: \emph{Notation abuse}: $ MH $ is class of graphs
  \item \textbf{Branch sets}: $ V_i $'s from above
  \item \textbf{Extended branch graph}: Branch set together with incident edges
  \item \textbf{Minor} ($ H $ of $ G $, noted $ H \preccurlyeq G $): $ \Leftrightarrow MH \subseteq G $ \\*
    $ \leadsto H \preccurlyeq G \Leftrightarrow H $ can be obtained by edge/vertex deletions + contractions.
  \item \textbf{Topological minor}: $ H $ is topological minor if $ TH \subseteq G $ where $ TH $ is built from $ H $ by subdividing edges
  \item \textbf{Note}: $ TH \subseteq MH $
\end{itemize}

\paragraph{Coloring}
\begin{itemize}
  \item \textbf{Co-clique number}: $ \alpha(G) = $ size of largest independent set
  \item \textbf{Clique number}: $ \omega(G) = $ size of largest clique
  \item \textbf{Proper coloring}: $ = c : V(G) \to [k] $ with $ c(u) \neq c(v) \quad (\forall uv \in E(G)) $
  \item \textbf{Equitable coloring}: proper coloring + color classes have almost ($ \pm 1 $) equal size
  \begin{itemize}
    \item \emph{existence}: any graph has equitable coloring in $ (\Delta(G) + 1) $ colors 
  \end{itemize}
  \item \textbf{4-color-theorem}: $ G $ planar $ \Rightarrow \chi(G) \leq 4 $ 
  \item \textbf{ij-flip}: $ c': V(G) \to [k] $ is $ ij $-flip at $ v \in V(G) $ \\*
    $ \Leftrightarrow c' $ obtained by flipping colors $ i $ and $ j $ in max. conn. component containing $ v $
  \item 
\end{itemize}

\paragraph{Chromatic number}
\begin{itemize}
  \item \textbf{Definition}: $ \chi(G) = \min\{ k: G \text{ has proper coloring with } k \text{ colors} \} $ 
  \item \textbf{Examples}: $ \chi(C_{2n}) = 2 $, $ \chi(C_{2n+1}) = 3 $
  \item \textbf{Properties}:
  \begin{itemize}
    \item $ \chi(G) \geq \omega(G) $ 
    \item $ \chi(G) \geq \frac{\vert G \vert}{\alpha(G)} $
    \item $ \chi(G) \leq \Delta(G) + 1 $ (\emph{greedy coloring})
    \item $ G $ connected, not complete, no odd cycles $ \Rightarrow \chi(G) \leq \Delta(G) $
  \end{itemize}
\end{itemize}

\paragraph{Perfect graph}
\begin{itemize}
  \item \textbf{Definition}: $ \forall H \underset{\text{ind}}{\subseteq} G : \chi(H) = \omega(H) $
  \item \textbf{Perfect complement}: $ G $ is perfect $ \Leftrightarrow \overline{G} $ is perfect
  \item \textbf{Perfect graph conjecture}: $ G $ is perfect $ \Leftrightarrow $
    \begin{equation*}
      C_{2k+1} \not \subseteq G \text{ for } k \geq 2 \wedge \overline{C_{2k+1}} \not \subseteq G
    \end{equation*}
\end{itemize}

\paragraph{Posets}
\begin{itemize}
  \item \textbf{Definition}: antisymmetric, reflexive, transitive relation on $ X $ \\* (write $ x \leq y $ instead of $ (x,y) $) 
  \item \textbf{Incidence poset} of $ G $: poset whose cover diagram is represented by $ IG $ with vertices all below the edges
  \item \textbf{Poset dimension}: $ \dim(R) = $ smallest $ k \in \N : R $ is intersection of $ k $ total orders
  \item \textbf{Poset dimension in planar graphs}: $ G $ planar $ \Leftrightarrow \dim(\text{incidence poset}) \leq 3 $
\end{itemize}

\paragraph{List-colorings}
\begin{itemize}
  \item \textbf{L-list-colorable}: if $ \exists \ c: V \to \N \ \forall v \in V : c(v) \in L(v) $ \\*
  (for \emph{list of colors} $ L(v) \subseteq \N $ for each vertex, adjacent vertices receive different colors) 
  \item \textbf{k-list-colorable/-choosable}: if $ G $ is $ L $-list-colorable for each list $ L $
  \item \textbf{List chromatic number}: $ \chi_l(G) = \text{ch}(G) $ \\* $ = \min\left\{ k : G \text{ is $ L $-colorable } \forall L: V \to 2^\N: \vert L(v) \vert = k \forall v \in V(G) \right\} $
  \begin{itemize}
    \item $ \chi_l(G) \geq \chi(G) $ because we can choose $ L(v) = \{ 1, \dots, k \} $ ($ \forall v \in V(G) $) 
    \item often $ \chi_l(G) \gg \chi(G) $ (see $ K_{m,n} $: $ \chi = 2 $, $ \chi_l \approx \log n $)
  \end{itemize}
  \item \textbf{Planar graphs}: $ \chi_l(G) \leq 5 $
  \item \textbf{Locally planar graphs}: $ \chi_l(G) \leq 5 $
\end{itemize}