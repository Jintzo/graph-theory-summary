\section{Coloring}

\paragraph{Base definitions}
\begin{itemize}
  \item \textbf{Vertex coloring}: map $ c: V(G) \to S $ with $ c(v) \neq c(w) $ for adjacent $ v,w $
  \begin{itemize}
    \item \emph{color set} $ S $
    \item \emph{k-coloring}: coloring $ c: V(G) \to S $ with $ \vert S \vert = k $
  \end{itemize}
  \item \textbf{(Vertex) chromatic number}: $ = \chi(G) \coloneqq \min\{ k \in \N : G \text{ has $ k $-coloring} \} $
  \begin{itemize}
    \item $ \chi(G) \geq \omega(G) $ 
    \item $ \chi(G) \geq \frac{\vert G \vert}{\alpha(G)} $
    \item $ \chi(G) \leq \Delta(G) + 1 $ (\emph{greedy coloring})
    \item $ G $ connected, not complete, no odd cycles $ \Rightarrow \chi(G) \leq \Delta(G) $
    \item \emph{k-chromatic} graph: $ \chi(G) = k $ 
    \item \emph{k-colorable} graph: $ \chi(G) \leq k $ 
  \end{itemize}
  \item \textbf{Color classes}: partitions of $ V(G) $ with same color
  \item \textbf{Equitable coloring}: proper coloring + color classes have almost ($ \pm 1 $) equal size
  \begin{itemize}
    \item \emph{existence}: any graph has equitable coloring in $ (\Delta(G) + 1) $ colors 
  \end{itemize}
  \item \textbf{ij-flip}: $ c': V(G) \to [k] $ is $ ij $-flip at $ v \in V(G) $ \\*
    $ \Leftrightarrow c' $ obtained by flipping colors $ i $ and $ j $ in max. conn. component containing $ v $
  \item \textbf{Edge coloring}: map $ c: E(G) \to S $ with $ c(e) \neq c(f) $ for adjacent $ e,f $
  \begin{itemize}
    \item edge coloring of $ G  \Leftrightarrow $ vertex coloring of $ L(G) $
    \item \emph{k-edge-coloring}: edge-coloring $ c: E(G) \to S $ with $ \vert S \vert = k $
  \end{itemize}
  \item \textbf{Edge chromatic number}: $ = \chi'(G) \coloneqq \min\{ k \in \N : G \text{ has $ k $-edge-coloring} \} $
\end{itemize}

\paragraph{Coloring maps and planar graphs}
\begin{itemize}
  \item \textbf{4-color-theorem}: every planar graph is 4-colorable
  \item \textbf{3-color-theorem}: every triangle-free planar graph is 3-colorable
\end{itemize}

\paragraph{Coloring vertices}
\begin{itemize}
  \item \textbf{Chromatic number upper bound}: $ \chi(G) \leq \frac{1}{2} + \sqrt{2\Vert G \Vert + \frac{1}{4}} $ 
  \item \textbf{Greedy coloring}: sort vertices $ v_1, \dots, v_n $, color them with the smallest possible color starting at $ v_1 $
  \begin{itemize}
    \item[$ \leadsto $] never uses more than $ \Delta(G) + 1 $ colors 
  \end{itemize}
  \item \textbf{coloring number} of graph $ G $: $ \text{col}(G) \coloneqq $ smallest $ k $ s.t. $ G $ has vertex enumeration where each vertex is preceded by $ < k $ neighbors
\end{itemize}

\paragraph{Coloring edges}
\begin{itemize}
  \item \textbf{Vizing's theorem}: for every graph $ G $, $ \chi'(G) \in \{ \Delta(G), \Delta(G) + 1 \} $
  \item \textbf{Bipartite graphs}: $ \chi'(G) = \Delta(G) $
\end{itemize}

\paragraph{List coloring}
\begin{itemize}
  \item \textbf{L-list-colorable}: if $ \exists \ c: V \to \N \ \forall v \in V : c(v) \in L(v) $ \\*
  (for \emph{list of colors} $ L(v) \subseteq \N $ for each vertex, adjacent vertices receive different colors) 
  \item \textbf{k-list-colorable/-choosable}: if $ G $ is $ L $-list-colorable for each list $ L $
  \item \textbf{List chromatic number}: $ \chi_l(G) = \text{ch}(G) $ \\* $ = \min\left\{ k : G \text{ is $ L $-colorable } \forall L: V \to 2^\N: \vert L(v) \vert = k \forall v \in V(G) \right\} $
  \begin{itemize}
    \item $ \chi_l(G) \geq \chi(G) $ because we can choose $ L(v) = \{ 1, \dots, k \} $ ($ \forall v \in V(G) $) 
    \item often $ \chi_l(G) \gg \chi(G) $ (see $ K_{m,n} $: $ \chi = 2 $, $ \chi_l \approx \log n $)
  \end{itemize}
  \item \textbf{Planar graphs}: $ \chi_l(G) \leq 5 $
  \item \textbf{Locally planar graphs}: $ \chi_l(G) \leq 5 $
\end{itemize}

\paragraph{Perfect graphs}
\begin{itemize}
  \item \textbf{Clique number} of graph $ G $: $ \omega(G) \coloneqq \max\{ k \in \N : K_k \subseteq G \} $
  \item \textbf{Independence number} of graph $ G $: size of largest independent vertex set
  \item \textbf{Perfect graph}: $ \forall H \underset{\text{ind}}{\subseteq} G : \chi(H) = \omega(H) $
  \item \textbf{Perfect complement}: $ G $ is perfect $ \Leftrightarrow \overline{G} $ is perfect
  \item \textbf{Perfect graph conjecture}: $ G $ is perfect $ \Leftrightarrow $
    \begin{equation*}
      C_{2k+1} \not \subseteq G \text{ for } k \geq 2 \wedge \overline{C_{2k+1}} \not \subseteq G
    \end{equation*}
\end{itemize}

\paragraph{Posets}
\begin{itemize}
  \item \textbf{Definition}: antisymmetric, reflexive, transitive relation on $ X $ \\* (write $ x \leq y $ instead of $ (x,y) $) 
  \item \textbf{Incidence poset} of $ G $: poset whose cover diagram is represented by $ IG $ with vertices all below the edges
  \item \textbf{Poset dimension}: $ \dim(R) = $ smallest $ k \in \N : R $ is intersection of $ k $ total orders
  \item \textbf{Poset dimension in planar graphs}: $ G $ planar $ \Leftrightarrow \dim(\text{incidence poset}) \leq 3 $
\end{itemize}
